\documentclass[a4paper, 12pt]{article}
\usepackage{graphicx}
\usepackage{mathtools}
\usepackage{color}
\usepackage[UTF8]{ctex}

\begin{document}
    {\huge\title{实验报告}}
    {\large\author{程传哲}}
    \date{\today}
    \maketitle
\section{学习成果}

\subsection{版本控制git}
\begin{enumerate}
  \item {\large git init NewDir} 
    \begin{itemize}
      \item 使用git init命令来初始化一个Git仓库 NewDir是想要建在的文件下的文件名
    \end{itemize}
  \item {\large git clone \textless url\textgreater [directory]}
    \begin{itemize}
      \item 使用git clone去git仓库拷贝项目,其中url为git仓库地址,directory为本地目录
    \end{itemize}
  \item {\large git config}
    \begin{itemize}
      \item 使用git config来配置用户名和邮箱地址例如git config global --user.name '程传哲'
    \end{itemize}
  \item{\large git add}
    \begin{itemize}
      \item 使用git add可以将工作区的文件放入暂存区中,例如git add file01.txt以及git add .可以将之前的所有未保存到暂存区的文件暂存
    \end{itemize}
  \item{\large git status}
    \begin{itemize}
      \item 使用git status可以去查看相关文件的状态
    \end{itemize}
  \item{\large git diff}
    \begin{itemize}
      \item 使用git diff查看相关文件的更新日志
        \begin{itemize}
          \item 查看已缓存的改动: git diff --cached
          \item 查看已缓存的与未缓存的所有改动:git diff HEAD
          \item 显示摘要而非整个 diff:git diff --stat
        \end{itemize}
    \end{itemize}
  \item{\large git commit}
    \begin{itemize}
      \item 使用git commit可以将处于暂存区的文件放入仓库中,git commit -m '提交的注释' ,或者是未被放入暂存区的文件可以直接使用git commit -am '提交的注释'去将他放入到仓库中
    \end{itemize}
  \item{\large git reset HEAD}
    \begin{itemize}
      \item 使用git reset HEAD可以取消已经暂存的文件例如git reset HEAD file01.txt 
    \end{itemize}
  \item{\large git rm}
    \begin{itemize}
      \item 使用git rm一个删除命令
        \begin{itemize}
          \item git rm \textless file \textgreater 从已跟踪文件清单中移除
          \item git rm -f \textless file \textgreater 是将修改并已经放入暂存区的文件移除
          \item git rm cached \textless file \textgreater 把文件从暂存区域移除,但仍保留在当前工作目录
          \item git rm –r * 删除整个子目录和文件(递归删除)
        \end{itemize}
    \end{itemize}
  \item{\large git mv}
    \begin{itemize}
      \item git mv file01.txt file02.txt,所以git mv的功能是将一个文件移动或者重命名还包括目录和软链接
    \end{itemize}
  \item{\large git branch}
    \begin{itemize}
      \item 查看分支
    \end{itemize}
  \item{\large git branch (branchname)}
    \begin{itemize}
      \item 创建分支并命名
    \end{itemize}
  \item {\large git checkout (branchname)}
    \begin{itemize}
      \item 切换分支
    \end{itemize}
  \item{\large git branch -d (branchname)}
    \begin{itemize}
      \item 删除分支
    \end{itemize}
  \item{\large git log}
    \begin{itemize}
      \item 使用git log 查看更新的内容标签
        \begin{itemize}
          \item -pretty=oneline 简短内容
          \item -reverse逆向查看
          \item –author查找指定用户的提交日志
          \item –since、–before、 --until、–after指定筛选日期
        \end{itemize}
    \end{itemize}
\end{enumerate}
\subsection{文档编写latex}
\begin{enumerate}
\item{\large 字体效果}
  \begin{itemize}
    \item 下划线
    \begin{center}
        \underline{underline,程传哲}
    \end{center}
    \item 字体改观
    \begin{center}
        \textit{impossibility}
    \end{center}
  \end{itemize}
\item{\large 字体颜色}
    \begin{itemize}
      \item 普通改变颜色
      \begin{center}
        \item {\color{blue}CHENGCHUAZHE}
      \end{center}
      \item 改变字体背景颜色
      \begin{center}
        \item \colorbox{blue}{\color{yellow}CHENGCHUAZHE}
      \end{center}
    \end{itemize}
\item{\large 表格}
    \begin{itemize}
      \item 示例一:
      \begin{tabular}{|r|l|}
        \hline
        1   & math  \\
        \hline 
        12  & english  \\
        \hline 
      \end{tabular}
      \item 示例二:
      \begin{tabular}{rc|}
      LATEX & latex \\
      \cline{1-1}
      CTEX  & ctex  \\
      \hline
      \end{tabular}
    \end{itemize}
\item {\large 数学公式}
    \begin{itemize}
        \item 示例一:\begin{equation}
                a =  b + c \\ 
              \end{equation}
        \item 示例二:
            \begin{equation}
                \sum_{i=1}^5 y^z = \beta
            \end{equation}
    \end{itemize}
\item {\large 特殊字符}
    \begin{itemize}
        \item \# \$ \% \^{} \& \_ \{ \} \~{}
    \end{itemize}
\end{enumerate}

\section{Github链接}


\section{学习感悟}
    在本次的系统工具开发实践课中我受益颇多,有关于latex与git项目的学习也还会继续,在latex学习过程中愈发的明白这样一个排版工具对一个好的论文的重要性,latex的
很多地方可以避免我们在使用word编写排版论文时候的出现的问题,比如字体,段落,符号等。在学习git中也弄清楚了基础指令的作用以及他们给我们带来的效果,以及git这样一个工具对项目开发的实际作用,版本更迭等。之宗旨学习了这么多知识后,会增加我们学习计算机课程的兴趣
\end{document}